\documentclass{beamer}      
\mode<presentation>{\usetheme{AnnArbor}}
\usecolortheme{whale}

\usepackage{mathptmx}
\usepackage{helvet}

\setbeamercolor{frametitle}{parent=subsection in head/foot}
\setbeamercolor{frametitle right}{parent=section in head/foot}

% shows outline in overlayareas
% taken from https://tex.stackexchange.com/questions/573822/show-outline-around-overlayarea

\makeatletter
\newif\ifbeamer@show@overlayarea
\pgfkeys{/beamer/overlayarea/.cd,show frame/.is if=beamer@show@overlayarea,
show frame/.default=true}
\beamer@show@overlayareafalse
\mode
<presentation>
{\renewenvironment{overlayarea}[3][]{%
  \pgfkeys{/beamer/overlayarea/.cd,#1}%
  \beamer@animht=#2\relax
  \beamer@animwd=#3\relax
  \setbox\beamer@areabox=\vbox to\beamer@animwd\bgroup
  \strut\begin{minipage}[t]{\beamer@animht}%
  % Make the minipage behave like the main part of the slide
  \normalfont
  \raggedright
  }
  {%
  \end{minipage}\endgraf\vfil
  \egroup
  \wd\beamer@areabox=\beamer@animht
  \ht\beamer@areabox=\beamer@animwd
  \dp\beamer@areabox=0pt %
  \ifbeamer@show@overlayarea
   \bgroup\fboxsep=0pt\relax
   \hspace*{-0.4pt}\vspace*{-0.4pt}%
   \fbox{\box\beamer@areabox}\hspace*{-0.4pt}\vspace*{-0.4pt}%
   \egroup
  \else
   \box\beamer@areabox
  \fi 
}}
\makeatother


\makeatletter
\pgfdeclarehorizontalshading[frametitle.bg,frametitle right.bg]{beamer@frametitleshade}{\paperheight}{%
    color(0pt)=(frametitle.bg);
    color(\paperwidth)=(frametitle right.bg)}

\AtBeginDocument{
    \pgfdeclareverticalshading{beamer@topshade}{\paperwidth}{%
        color(0pt)=(bg);
        color(4pt)=(black!50!bg)}
}

\addtobeamertemplate{headline}
{}
{%
    \vskip-0.2pt
    \pgfuseshading{beamer@topshade}
    \vskip-2pt
}


\setbeamertemplate{frametitle}
{%
    \nointerlineskip%
    \vskip-2pt%
    \hbox{\leavevmode
        \advance\beamer@leftmargin by -12bp%
        \advance\beamer@rightmargin by -12bp%
        \beamer@tempdim=\textwidth%
        \advance\beamer@tempdim by \beamer@leftmargin%
        \advance\beamer@tempdim by \beamer@rightmargin%
        \hskip-\Gm@lmargin\hbox{%
            \setbox\beamer@tempbox=\hbox{\begin{minipage}[b]{\paperwidth}%
                    \vbox{}\vskip-.75ex%
                    \leftskip0.3cm%
                    \rightskip0.3cm plus1fil\leavevmode
                    \insertframetitle%
                    \ifx\insertframesubtitle\@empty%
                    \strut\par%
                    \else
                    \par{\usebeamerfont*{framesubtitle}{\usebeamercolor[fg]{framesubtitle}\insertframesubtitle}\strut\par}%
                    \fi%
                    \nointerlineskip
                    \vbox{}%
                \end{minipage}}%
                \beamer@tempdim=\ht\beamer@tempbox%
                \advance\beamer@tempdim by 2pt%
                \begin{pgfpicture}{0pt}{0pt}{\paperwidth}{\beamer@tempdim}
                    \usebeamercolor{frametitle right}
                    \pgfpathrectangle{\pgfpointorigin}{\pgfpoint{\paperwidth}{\beamer@tempdim}}
                    \pgfusepath{clip}
                    \pgftext[left,base]{\pgfuseshading{beamer@frametitleshade}}
                \end{pgfpicture}
                \hskip-\paperwidth%
                \box\beamer@tempbox%
            }%
            \hskip-\Gm@rmargin%
        }%
        \nointerlineskip
        \vskip-0.2pt
        \hbox to\textwidth{\hskip-\Gm@lmargin\pgfuseshading{beamer@topshade}\hskip-\Gm@rmargin}
        \vskip-2pt
    }
\makeatother

\setbeamercolor{section in toc}{fg=teal}
%%%
\definecolor{brightpink}{rgb}{1.0, 0.0, 0.5}
\setbeamercolor{structure}{fg=cyan!80!black}
\setbeamercolor*{block title example}{fg=blue!50,bg= blue!10}
\setbeamercolor*{block body example}{fg= red,bg= blue!5}
\usefonttheme{serif}

% \setbeamertemplate{footline}[page number]{} will show only page number in the footer

\setbeamertemplate{footline}{}

\setbeamercolor{headline}{fg=blue!90!black,bg=cyan!90!black}
\setbeamercolor{palette primary}{fg=white,bg=cyan!90!black}
\setbeamercolor{palette secondary}{fg=white,bg=cyan!90!black}
\setbeamercolor{palette tertiary}{fg=white,bg=black}
\setbeamercolor{frametitle}{fg=white,bg=cyan!50!black}
\setbeamercolor{alerted text}{fg=brightpink}


\colorlet{titleleft}{cyan}
\colorlet{titleright}{black}

\setbeamercolor*{frametitle}{fg=white}

\makeatletter
\pgfdeclarehorizontalshading[titleleft,titleright]{beamer@frametitleshade}{\paperheight}{%
    color(0pt)=(titleleft);
    color(\paperwidth)=(titleright)}
\makeatother

%============================ continue enumerate ===================================
% \setbeamercovered{highly dynamic}

\newcounter{saveenumi}
\newcommand{\seti}{\setcounter{saveenumi}{\value{enumi}}}
\newcommand{\conti}{\setcounter{enumi}{\value{saveenumi}}}

%====================================================================================

% ====================================================================================================

% =================== Different colors for different blocks =======================================
% Thanks to https://tex.stackexchange.com/questions/347269/different-colors-for-different-types-of-blocks-in-beamer

\setbeamercolor{block title}{use=structure,fg=structure.fg,bg=structure.fg!20!bg}
% \setbeamercolor{block body}{parent=normal text,use=block title,bg=block title.bg!50!bg}

\setbeamercolor{block title example}{use=example text,fg=example text.fg,bg=example text.fg!20!bg}
% \setbeamercolor{block body example}{parent=normal text,use=block title example,bg=block title example.bg!50!bg}

\addtobeamertemplate{proof begin}{%
    \setbeamercolor{block title}{fg=black,bg=yellow!50!white}
    % \setbeamercolor{block body}{fg=red, bg=red!30!white}
}{}

\BeforeBeginEnvironment{theorem}{
    \setbeamercolor{block title}{fg=black,bg=orange!50!white}
    % \setbeamercolor{block body}{fg=orange, bg=orange!30!white}
}
\AfterEndEnvironment{theorem}{
 \setbeamercolor{block title}{use=structure,fg=structure.fg,bg=structure.fg!20!bg}
 % \setbeamercolor{block body}{parent=normal text,use=block title,bg=block title.bg!50!bg, fg=black}
}

\BeforeBeginEnvironment{definition}{%
    \setbeamercolor{block title}{fg=black,bg=cyan!50!white}
    % \setbeamercolor{block body}{fg=pink, bg=pink!30!white}
}
\AfterEndEnvironment{definition}{
 \setbeamercolor{block title}{use=structure,fg=structure.fg,bg=structure.fg!20!bg}
 % \setbeamercolor{block body}{parent=normal text,use=block title,bg=block title.bg!50!bg, fg=black}
}

\BeforeBeginEnvironment{remark}{%
    \setbeamercolor{block title}{fg=black,bg=teal!50!white}
    % \setbeamercolor{block body}{fg=pink, bg=pink!30!white}
}
\AfterEndEnvironment{remark}{
 \setbeamercolor{block title}{use=structure,fg=structure.fg,bg=structure.fg!20!bg}
 % \setbeamercolor{block body}{parent=normal text,use=block title,bg=block title.bg!50!bg, fg=black}
}
% ====================================================================================================


\newcommand{\rbb}{\ensuremath{\mathbb{R}}}
\newcommand{\qbb}{\ensuremath{\mathbb{Q}}}
\newcommand{\zbb}{\ensuremath{\mathbb{Z}}}
\newcommand{\nbb}{\ensuremath{\mathbb{N}}}
\newcommand{\vectorsall}{\ensuremath{\mathbf{v}_1,\mathbf{v}_2,\ldots,\mathbf{v}_n}} 
\newcommand{\vectorsallw}{\ensuremath{\mathbf{w}_1,\mathbf{w}_2,\ldots,\mathbf{w}_n}} 
\newcommand{\xbf}{\ensuremath{\mathbf{x}}} 
\newcommand{\ybf}{\ensuremath{\mathbf{y}}} 
\newcommand{\zbf}{\ensuremath{\mathbf{z}}} 
\newcommand{\vbf}{\ensuremath{\mathbf{v}}}
\renewcommand{\bf}[1]{\ensuremath{\mathbf{#1}}}
\newcommand{\sbb}{\ensuremath{\mathbb{S}}}
\newcommand{\cali}[1]{\ensuremath{\mathcal{#1}}}
\newcommand{\scr}[1]{\ensuremath{\mathscr{#1}}}
\newcommand{\mscr}{\ensuremath{\mathscr{M}}}
\newcommand{\nscr}{\ensuremath{\mathscr{N}}}
\renewcommand{\sf}[1]{\ensuremath{\mathsf{#1}}}
\newcommand{\red}[1]{\textcolor{red}{#1}} 
\newcommand{\blue}[1]{\textcolor{blue}{#1}} 
\newcommand{\brown}[1]{\textcolor{brown}{#1}} 
\newcommand{\magenta}[1]{\textcolor{magenta}{#1}}
\definecolor{defn}{rgb}{0.9, 0.17, 0.31}
\newcommand{\defn}[1]{\textcolor{defn}{#1}}
\newcommand{\cyan}[1]{\textcolor{cyan}{#1}}
\newcommand{\bb}[1]{\ensuremath{\mathbb{#1}}} 
\newcommand{\delbydel}[2]{\ensuremath{\dfrac{\partial#1}{\partial#2}}} 
\newcommand{\crf}{\ensuremath{\mathrm{Cr}(f)}} 
\newcommand{\grad}{\ensuremath{\nabla}} 
\newcommand{\dbb}{\ensuremath{\mathbb{D}}} 
\newcommand{\dist}{\ensuremath{\operatorname{dist}}} 
\newcommand{\dbyd}[2]{\ensuremath{\dfrac{d#1}{d#2}}} 
\newcommand{\isom}{\ensuremath{\cong}} 
\renewcommand{\cal}[1]{\ensuremath{\mathcal{#1}}} 
\newcommand{\acal}{\ensuremath{\cal{A}}}
\newcommand{\im}{\ensuremath{\operatorname{im}}}
\newcommand{\defeq}{\vcentcolon=}
\newcommand{\eqdef}{=\vcentcolon}
\newcommand{\define}{\overset{\mathrm{def}}{=\joinrel=}}
\newcommand{\tensor}{\ensuremath{\otimes}}
\newcommand{\directsum}{\ensuremath{\oplus}}
\newcommand{\homeo}{\ensuremath{\cong}}
\newcommand{\cutn}{\ensuremath{\mathrm{Cu}(N)}}
\newcommand{\sen}{\ensuremath{\mathrm{Se}(N)}}
\newcommand{\innerprod}[2]{\ensuremath{\left\langle #1,#2\right\rangle}}
\newcommand{\norm}[1]{\ensuremath{\left\|#1\right\|}}
\newcommand{\range}{\ensuremath{\mathrm{range}}}
\newcommand{\paran}[1]{\ensuremath{\left( #1 \right)}}
\newcommand{\curlybracket}[1]{\ensuremath{\left\{ #1 \right\}}}
\newcommand{\squarebracket}[1]{\ensuremath{\left[ #1 \right]}}
\newcommand{\abs}[1]{\ensuremath{\left|#1\right|}}
\newcommand{\aTransInverse}{\ensuremath{\paran{A^T}^{-1}}}
\newcommand{\sqrtATransAInverse}{\ensuremath{\paran{\sqrt{A^TA}}^{-1}}}
\newcommand{\trace}[1]{\ensuremath{\mathrm{tr}\left( #1 \right)}}
\newcommand{\cu}{\ensuremath{\mathrm{Cu}(p)~}}
\newcommand{\se}{\ensuremath{\mathrm{Se}(p)~}} 
\newcommand{\co}{\ensuremath{\mathrm{Co}(x_0,\delta)~}} 
\newcommand{\costar}{\ensuremath{\mathrm{Co}^\star(x_0,\delta)~}}
\newcommand{\Ball}{\ensuremath{\overline{B(x_0,\delta)}~}}
\newcommand{\scal}{\mathcal{S}}
\newcommand{\bcal}{\mathcal{B}}
\newcommand{\hess}{\mathrm{Hess}}

% New commands for todo notes

\newcommand{\lub}{\ensuremath{\mathrm{lub}}}
\theoremstyle{definition} 
\newtheorem*{remark}{Remark}
\newtheorem*{coro}{Corollary}
\newtheorem*{prop}{Proposition}
\newcommand{\p}{\pause}

% Newcommands fo r font size
\newcommand{\fontsmall}{\fontsize{7pt}{8.2}\selectfont}


\usepackage{amsmath, amssymb, amscd, amsthm, amsfonts, amsbsy, appendix}
\usepackage{lmodern}
\usepackage{bbm, mathrsfs, bm, linearb, upgreek, textcomp, mathdots}
\usepackage{mathtools, url}
\usepackage[all, cmtip]{xy} 
\usepackage[intoc]{nomencl}
\usepackage{pgf}
\usepackage{calligra}
\usepackage{ebgaramond}
\usepackage{bookmark}% faster updated bookmarks
\usepackage[most]{tcolorbox}
\usepackage{varwidth}   %% provides varwidth environment

\usepackage{tikz,tcolorbox}
\usetikzlibrary{cd}
\usetikzlibrary{shapes.callouts}

\usepackage{etoolbox}

\makeatletter
\newcommand{\DrawLine}{%
  \begin{tikzpicture}
  \path[use as bounding box] (0,0) -- (\linewidth,0);
  \draw[color=red!50!yellow,dashed,dash phase=2pt]
        (0-\kvtcb@leftlower-\kvtcb@boxsep,0)--
        (\linewidth+\kvtcb@rightlower+\kvtcb@boxsep,0);
  \end{tikzpicture}%
  }
\makeatother

% For pause in align
%https://tex.stackexchange.com/questions/16186/equation-numbering-problems-in-amsmath-environments-with-pause/75550#75550
\makeatletter
\let\save@measuring@true\measuring@true
\def\measuring@true{%
  \save@measuring@true
  \def\beamer@sortzero##1{\beamer@ifnextcharospec{\beamer@sortzeroread{##1}}{}}%
  \def\beamer@sortzeroread##1<##2>{}%
  \def\beamer@finalnospec{}%
}

% For continued proof 

\newenvironment<>{proofs}[1][\proofname]{%
    \par
    \def\insertproofname{#1\@addpunct{.}}%
    \usebeamertemplate{proof begin}#2}
  {\usebeamertemplate{proof end}}
\makeatother

% \hypersetup{pdfpagemode=FullScreen}




\AtBeginSection[]{
  \begin{frame}
  \vfill
  \centering
  \begin{beamercolorbox}[sep=8pt,center,shadow=true,rounded=true]{title}
    \usebeamerfont{title}\secname\par%
  \end{beamercolorbox}
  \vfill
  \end{frame}
}

%\hypersetup{pdfpagemode=FullScreen}
\title[Cut Locus]{Equivariant Cut Locus Theorem \\[1ex] \small Inter IISER-NISER Mathematics Meet, 2022\\[1ex] 
\small IISER Kolkata }
\author{Sachchidanand Prasad}
\institute[IISERK]{\small Indian Institute of Science Education and Research Kolkata}
\date{1st June 2022}
% logo 
\titlegraphic{%
   \includegraphics[width=1.5cm,keepaspectratio]{figures/iiserk-logo.png}%
}

% \logo{%
%   \makebox[0.95\paperwidth]{%
%     % \includegraphics[width=4cm,keepaspectratio]{img/csm_dmv-logo.png}%
%     \hfill%
%     \includegraphics[width=1.5cm,keepaspectratio]{img/iiserk-logo.png}%
%   }%
% }

% \definecolor{titlepage}{rgb}{0.98, 0.94, 0.75}

\setbeamertemplate{navigation symbols}{} % Navigation symbols will be removed
% \setbeamertemplate{footline}{}
\setbeamertemplate{itemize items}[ball]  % Itemize into balls


\begin{document}
	%\setbeamercolor{background canvas}{bg=titlepage}
	\titlepage
	% \vspace{0.1cm}
	% \begin{center}
	% 	Indian Institute of Science Education and Research, Kolkata
	% \end{center}

	\begin{frame}
		\frametitle<presentation>{Outline of the talk}
		\tableofcontents
	\end{frame}	

	\section{Statement of the main result}
	\begin{frame}
		\frametitle<presentation>{Cut locus of a point}

		\p 
		\begin{definition}[Cut locus] \label{defn:cut_locus_of_a_point}
			\p Let $M$ be a Riemannian manifold \p and $p$ be any point in $M$. \p If $\mathrm{Cu}(p) $ denotes the \emph{\defn{cut locus of $p$}}, \p then we say that $q\in \mathrm{Cu}(p)  $ if there exists a distance minimal geodesic joining $p$ to $q$ \p such that any extension of it beyond $q$ is not a distance minimal geodesic.
		\end{definition}
	\end{frame}

	\begin{frame}
		\frametitle<presentation>{Examples: $\mathbb{S}^2$}
		\begin{figure}[htbp]
			\begin{overlayarea}{6cm}{6cm}
				\includegraphics<2>[page = 1, width = \textwidth]{figures/sphere-cut_locus_point.pdf}
				\includegraphics<3>[page = 2, width = \textwidth]{figures/sphere-cut_locus_point.pdf}
				\includegraphics<4>[page = 3, width = \textwidth]{figures/sphere-cut_locus_point.pdf}
				\includegraphics<5>[page = 4, width = \textwidth]{figures/sphere-cut_locus_point.pdf}
				\includegraphics<6>[page = 5, width = \textwidth]{figures/sphere-cut_locus_point.pdf}
			\end{overlayarea}
		\end{figure}
	\end{frame}

	\begin{frame}
		\frametitle<presentation>{Examples: Torus}
		\begin{figure}[htbp]
			\centering
			\begin{overlayarea}{8cm}{6cm}
				\includegraphics<2>[page = 1, width = \textwidth]{figures/torus-cut-locus.pdf}
				\includegraphics<3>[page = 2, width = \textwidth]{figures/torus-cut-locus.pdf}
				\includegraphics<4>[page = 3, width = \textwidth]{figures/torus-cut-locus.pdf}
				\includegraphics<5>[page = 4, width = \textwidth]{figures/torus-cut-locus.pdf}
			\end{overlayarea}
		\end{figure}
	\end{frame}

	\begin{frame}
		\frametitle<presentation>{Cut locus of  a submanifold}

		\p
		\begin{definition}[Distance minimal geodesic] \label{defn:distance_minimal_geodesic}
			\p A geodesic $\gamma $ is called a \emph{\defn{distance minimal geodesic}} joining $N$ to $p$ \p if there exists $q\in N$ \p such that $\gamma$ is a minimal geodesic joining $q$ to $p$ \p and $l(\gamma)= d(p,N) $. \p We will call such geodesics as \textit{$N$-geodesics}.
		\end{definition}

		\p 
		\begin{definition}[Cut locus of a submanifold] \label{defn:cut_locus}
			\p Let $M$ be a Riemannian manifold \p and $N$ be any non-empty subset of $M$. \p If $\cutn$ denotes the \emph{\defn{cut locus of $N$}}, \p then we say that $q\in \cutn $ if there exists an $N$-geodesic joining $N$ to $q$ \p such that any extension of it beyond $q$ is not a distance minimal geodesic.
		\end{definition}
	\end{frame}

	\begin{frame}
		\frametitle<presentation>{An Example}
		\begin{figure}[htbp]
			\begin{overlayarea}{6cm}{6cm}
				\includegraphics<2>[page = 1, width = \textwidth]{figures/sphere-cut-locus-submanifold.pdf}
				\includegraphics<3>[page = 2, width = \textwidth]{figures/sphere-cut-locus-submanifold.pdf}
				\includegraphics<4>[page = 3, width = \textwidth]{figures/sphere-cut-locus-submanifold.pdf}
				\includegraphics<5>[page = 4, width = \textwidth]{figures/sphere-cut-locus-submanifold.pdf}
				\includegraphics<6>[page = 5, width = \textwidth]{figures/sphere-cut-locus-submanifold.pdf}
			\end{overlayarea}
		\end{figure}
	\end{frame}	
	

	\begin{frame}
		\frametitle<presentation>{Separating set of \texorpdfstring{$N$}{N}}

		\p 
		\begin{definition}[Separating set] \label{defn:separating_set}
			\p Let $N$ be a subset of a Riemannian manifold $M$. \p The \emph{\defn{separating set}}, denoted by $\sen$, \p  consists of all points $q\in M$ \p such that at least two distance minimal geodesics from $N$ to $q$ exist.
		\end{definition}

		\p 
		\begin{theorem}\label{thm:se_closure_is_cu}
			For a complete Riemannian manifold $M$ and a compact submanifold $N$ of $M$, 
			\begin{displaymath}
				\overline{\sen} = \cutn.
			\end{displaymath}
		\end{theorem}
	\end{frame}

	\begin{frame}
		\frametitle<presentation>{Statement of the main theorem}
		\p
		\begin{theorem}\label{thm:main_theorem}
			\p Let $M$ be a complete, \p closed \p and connected Riemannian manifold \p and $G$ be any compact Lie group which acts on $M$ \alert<6>{freely} and \alert<7>{isometrically}. \p[8]Let $N$ be any $G$-invariant closed submanifold of $M$, \p[9] then
			\begin{displaymath}
				\cutng ``=" \cutn/G	
			\end{displaymath}
		\end{theorem}
	\end{frame}	

	\begin{frame}
		\begin{itemize}
			\p \item We will prove the theorem for $\textcolor[HTML]{088ec4}{\mathrm{Se}}$, i.e., \p
		\begin{displaymath}
			\textcolor[HTML]{088ec4}{\mathrm{Se}} \left(N/G\right) = \textcolor[HTML]{088ec4}{\mathrm{Se}}(N)/G.
		\end{displaymath}
		\p \item Since the action is isometric, $\textcolor[HTML]{088ec4}{\mathrm{Se}}(N)$ is $G$-invariant. \p Now in order to show that $\overline{\textcolor[HTML]{088ec4}{\mathrm{Se}}(N)}$ is $G$-invariant, \p let $x\in \overline{\textcolor[HTML]{088ec4}{\mathrm{Se}}(N)}$. \p So there exists a sequence $(x_n)\subset \textcolor[HTML]{088ec4}{\mathrm{Se}}(N)$ such that $x_n\to x$, \p which implies $g\cdot x_n\to g\cdot x$. \p Hence, $g\cdot x \in \overline{\textcolor[HTML]{088ec4}{\mathrm{Se}}(N)}$.
		\end{itemize}
	\end{frame}	

	\section{Idea of the proof}
	\begin{frame}
		\frametitle{Idea of the proof}
		\begin{columns}[T] % align columns
			\begin{column}{.44\textwidth}
			\begin{overlayarea}{\textwidth}{\textheight}
				\includegraphics<2>[page = 1,  scale=0.45]{figures/se_of_N_mod_G.pdf}
					\includegraphics<3>[page = 2,  scale=0.45]{figures/se_of_N_mod_G.pdf}
					\includegraphics<4>[page = 3,  scale=0.45]{figures/se_of_N_mod_G.pdf}
					\includegraphics<5>[page = 4,  scale=0.45]{figures/se_of_N_mod_G.pdf}
					\includegraphics<6->[page = 5, scale=0.45]{figures/se_of_N_mod_G.pdf}
			\end{overlayarea}
			\end{column}%
			\hfill% 
			\begin{column}{.52\textwidth}
				\vspace{1.5cm}
				\begin{center}
					\p[7]\Large Problems in the approach
				\end{center}
				\begin{enumerate}
					\p[8]\item \alert<8>{Why is $(\pi\circ \gamma)$ a distance minimal geodesic?}
					\p[9]\item \alert<9>{Why are $(\pi\circ \gamma)$ and $(\pi\circ \eta)$ distinct?}
					\p[10] \item \alert<10>{The same for the lifts}.
				\end{enumerate}
			\end{column}%
		\end{columns}
	\end{frame}

	
	\begin{frame}
		\frametitle<presentation>{Connection on principal bundle}
		\p
		\begin{definition}[Ehresmann connection] \label{defn:ehresmann-connection}
			\p Given a smooth principal $G$-bundle $\pi:E\to B$, \p an \emph{\defn{Ehresmann connection}} on $E$ is a smooth subbundle $\mathcal{H}$ of $TE$, called the \emph{\defn{horizontal bundle}} of the connection, \p such that $TE=\mathcal{H}\oplus \mathcal{V}$, \p where $\mathcal{V}_p=\ker \left(d\pi_p:T_pE\to T_{\pi(p)}B\right)$.
		\end{definition}
		\begin{itemize}
			\p \item The bundle $\mathcal{V}$ is called the \emph{\defn{vertical bundle}} \p and it is independent of the connection chosen.
			\p \item $d\pi$ restricts to $\mathcal{H}_p$ is an isomorphism on $T_{\pi(p)}B$.
			\p \item $dg$ maps $\mathcal{H}_p$ to $\mathcal{H}_{g\cdot p} $. 
		\end{itemize}
	\end{frame}	

	\begin{frame}
		\frametitle<presentation>{Horizontal Lift}
		\p
		\begin{definition}[Horizontal lift]\label{defn:horizontal-lift}
			\p Let $\pi:E\to B$ be a fibre bundle with a connection $\mathcal{H}$. \p Let $\gamma$ be a smooth curve in $B$ through $\gamma(0)= b$. \p Let $e\in E$ be such that $\pi(e)=b$. \p A \emph{\defn{horizontal lift}} of $\gamma$ through $e$ is a curve $\tilde{\gamma}$ in $E$ \p such that $\pi\circ \tilde{\gamma}=\gamma,~\tilde{\gamma}(0)= e$, \p and $\tilde{\gamma}'(t)\in \mathcal{H}_{\tilde{\gamma}(t)}$.
		\end{definition}

		\p 
		\begin{prop}[Uniqueness of horizontal lift]
			\p If $\gamma:[-1,1]\to B$ is a smooth curve such that $\gamma(0)=b$ and $e_0 \in \pi^{-1}(b)$, \p then there is a unique horizontal lift $\tilde{\gamma}$ through $e_0\in E$.
		\end{prop}
	\end{frame}	
	

	\section{Geodesics on \texorpdfstring{$M$}{M} and \texorpdfstring{$M/G$}{M/G}}
	\begin{frame}
		\frametitle<presentation>{Geodesic on $M$ and $M/G$}
		\p 
		\begin{theorem}
			There is a one-to-one correspondence between the geodesics on $M/G$ and geodesics on $M$ which are horizontal.
		\end{theorem}
	\end{frame}

	\begin{frame}
		\frametitle<presentation>{Proof of the correspondence}
		\p 
		 Let $\pi:E\to B$ be a Riemannian submersion between Riemannian manifolds $E$ and $B$.
		 \p 
		\begin{lemma}[O'Neil]
        	If $\tilde{\gamma}$ is a geodesic on $E$ and $\tilde{\gamma}'(0)\in \mathcal{H}_{\gamma(0)}$, \p then for all $t$, $\gamma'(t)\in \mathcal{H}_{\gamma(t)}$ \p and $\pi\circ \gamma$ is a geodesic on $B$. Moreover, the length is preserved.
        \end{lemma}


		 \p\textcolor{red}{Does the horizontal lift $\tilde{\gamma}$ of a geodesic $\gamma$ become a geodesic?}

		\p
		\begin{lemma}
			Let $\gamma$ be a geodesic in $B$. If $\tilde{\gamma}$ is the horizontal lift of $\gamma$, then
			\begin{itemize}
				\p \item $l \left(\gamma\right)=l \left(\tilde{\gamma}\right)$.\tabularnewline
				\p \item $\tilde{\gamma}'(t)\in \mathcal{H}_{\gamma(t)}$
				\p \item $\tilde{\gamma}$ is a geodesic in $E$.
			\end{itemize}
		\end{lemma}
		
        \p Since $\pi:M\to M/G$ is Riemannian submersion, \p we will get a one-to-one correspondence between the geodesics on $M/G$ and geodesics on $M$ which are horizontal.
	\end{frame}	


	\section{Proof of the main theorem}

	\begin{frame}
		\frametitle<presentation>{$\mathrm{Se} \left(N\right)/G \subseteq \mathrm{Se}\left(N/G\right)$}
		\begin{columns}[T] % align columns
			\begin{column}{.44\textwidth}
			\vspace{2cm}
			\begin{overlayarea}{\textwidth}{\textheight}
				\includegraphics<2>[page = 1, scale=0.45]{figures/proof_of_the_main_theorem-one-side.pdf}
				\includegraphics<3-7>[page = 2, scale=0.45]{figures/proof_of_the_main_theorem-one-side.pdf}
				\includegraphics<8-10>[page = 3, scale=0.45]{figures/proof_of_the_main_theorem-one-side.pdf}
				\includegraphics<11>[page = 4, scale=0.45]{figures/proof_of_the_main_theorem-one-side.pdf}
				\includegraphics<12-13>[page = 5, scale=0.45]{figures/proof_of_the_main_theorem-one-side.pdf}
				\includegraphics<14>[page = 6, scale=0.45]{figures/proof_of_the_main_theorem-one-side.pdf}
				\includegraphics<15->[page = 7, scale=0.45]{figures/proof_of_the_main_theorem-one-side.pdf}
			\end{overlayarea}
			\end{column}%
			\hfill% 
			\vspace{2cm}
			\begin{column}{.52\textwidth}
				\begin{itemize}
					\p[4] \item \small $\tilde{\gamma}$ is a geodesic and $l \left(\tilde{\gamma}\right)=d \left(\tilde{p},N\right)$ \p[5] $\implies \tilde{\gamma}'(1)\in \mathcal{H}_{\tilde{\gamma}(1)}$ \p[6] $\implies  \tilde{\gamma}'(t)\in \mathcal{H}_{\tilde{\gamma}(t)}$ \p[7] $\implies \tilde{\gamma}$ is a horizontal geodesic \p[9] $\implies \gamma$ is a geodesic.
					\p[10] \item  If $\tilde{p}\in \mathrm{Se}(N)$, \p[11] then there exists two $N$-geodesic, say $\tilde{\gamma}$ and $\tilde{\eta}$. \p[12] Due to uniqueness of horizontal lift, both will project to distinct geodesic and lengths are same. 
					\p[13] Note that $\tilde{\gamma}$ is an $N$-geodesic implies $\gamma$ will be an $N/G$-geodesic. \p[14]Otherwise, $\exists~\delta$ which is $N/G$ geodesic joining $p$ to $N/G$ \p[15] which gives a horizontal lift $\tilde{\delta}$ whose length is strictly less than $\tilde{\gamma}$, \p[16] a contradiction.
				\end{itemize}
			\end{column}%
		\end{columns}
	\end{frame}	

	\begin{frame}
		\frametitle<presentation>{$\mathrm{Se} \left(N\right)/G \supseteq \mathrm{Se}\left(N/G\right)$}
		\begin{columns}[T] % align columns
			\begin{column}{.44\textwidth}
			\vspace{2cm}
			\begin{overlayarea}{\textwidth}{\textheight}
				\includegraphics<2>[page = 1, scale=0.45]{figures/proof_of_the_main_theorem-other-side.pdf}
				\includegraphics<3>[page = 2, scale=0.45]{figures/proof_of_the_main_theorem-other-side.pdf}
				\includegraphics<4>[page = 3, scale=0.45]{figures/proof_of_the_main_theorem-other-side.pdf}
				\includegraphics<5>[page = 4, scale=0.45]{figures/proof_of_the_main_theorem-other-side.pdf}
				\includegraphics<6->[page = 5, scale=0.45]{figures/proof_of_the_main_theorem-other-side.pdf}
			\end{overlayarea}
			\end{column}%
			\hfill% 
			\vspace{2cm}
			\begin{column}{.52\textwidth}
				\begin{itemize}
					\p[3] \item \small If $\gamma$ is an $N/G$-geodesic starting from $p$, \p[4] then its horizontal lift $\tilde{\gamma}$ will be an $N$-geodesic. \p[5] If not, let $\tilde{\eta}$ be such that $l \left(\tilde{\eta}\right)=d \left(\tilde{p},N\right)$ which implies $\tilde{\eta}$ is horizontal. \p[6] Hence, $\eta=\pi\circ \tilde{\eta}$ will be a geodesic and 
					\begin{align*}
						l(\gamma) & = \p[7] d \left(p,N/G\right) = l(\eta) \\ 
						\p[8] & = l \left(\tilde{\eta}\right) \p[9] < l \left(\tilde{\gamma}\right) \p[10] = l(\gamma),
					\end{align*}
					\p[11] a contradiction. 
					\p[12] \item Thus, $\tilde{p}\in \mathrm{Se}(N)$.
				\end{itemize}
			\end{column}%
		\end{columns}
	\end{frame}

	\begin{frame}
		\begin{center}
			\Huge Thank you for your attention!
		\end{center}
	\end{frame}

\end{document}
