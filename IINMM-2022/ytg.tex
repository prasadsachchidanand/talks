\documentclass{beamer}      
\mode<presentation>{\usetheme{AnnArbor}}
\usecolortheme{whale}

\usepackage{mathptmx}
\usepackage{helvet}

\setbeamercolor{frametitle}{parent=subsection in head/foot}
\setbeamercolor{frametitle right}{parent=section in head/foot}

\makeatletter
\pgfdeclarehorizontalshading[frametitle.bg,frametitle right.bg]{beamer@frametitleshade}{\paperheight}{%
    color(0pt)=(frametitle.bg);
    color(\paperwidth)=(frametitle right.bg)}

\AtBeginDocument{
    \pgfdeclareverticalshading{beamer@topshade}{\paperwidth}{%
        color(0pt)=(bg);
        color(4pt)=(black!50!bg)}
}

\addtobeamertemplate{headline}
{}
{%
    \vskip-0.2pt
    \pgfuseshading{beamer@topshade}
    \vskip-2pt
}


\setbeamertemplate{frametitle}
{%
    \nointerlineskip%
    \vskip-2pt%
    \hbox{\leavevmode
        \advance\beamer@leftmargin by -12bp%
        \advance\beamer@rightmargin by -12bp%
        \beamer@tempdim=\textwidth%
        \advance\beamer@tempdim by \beamer@leftmargin%
        \advance\beamer@tempdim by \beamer@rightmargin%
        \hskip-\Gm@lmargin\hbox{%
            \setbox\beamer@tempbox=\hbox{\begin{minipage}[b]{\paperwidth}%
                    \vbox{}\vskip-.75ex%
                    \leftskip0.3cm%
                    \rightskip0.3cm plus1fil\leavevmode
                    \insertframetitle%
                    \ifx\insertframesubtitle\@empty%
                    \strut\par%
                    \else
                    \par{\usebeamerfont*{framesubtitle}{\usebeamercolor[fg]{framesubtitle}\insertframesubtitle}\strut\par}%
                    \fi%
                    \nointerlineskip
                    \vbox{}%
                \end{minipage}}%
                \beamer@tempdim=\ht\beamer@tempbox%
                \advance\beamer@tempdim by 2pt%
                \begin{pgfpicture}{0pt}{0pt}{\paperwidth}{\beamer@tempdim}
                    \usebeamercolor{frametitle right}
                    \pgfpathrectangle{\pgfpointorigin}{\pgfpoint{\paperwidth}{\beamer@tempdim}}
                    \pgfusepath{clip}
                    \pgftext[left,base]{\pgfuseshading{beamer@frametitleshade}}
                \end{pgfpicture}
                \hskip-\paperwidth%
                \box\beamer@tempbox%
            }%
            \hskip-\Gm@rmargin%
        }%
        \nointerlineskip
        \vskip-0.2pt
        \hbox to\textwidth{\hskip-\Gm@lmargin\pgfuseshading{beamer@topshade}\hskip-\Gm@rmargin}
        \vskip-2pt
    }
\makeatother

\setbeamercolor{section in toc}{fg=teal}
%%%
\definecolor{brightpink}{rgb}{1.0, 0.0, 0.5}
\setbeamercolor{structure}{fg=cyan!80!black}
\setbeamercolor*{block title example}{fg=blue!50,bg= blue!10}
\setbeamercolor*{block body example}{fg= red,bg= blue!5}
\usefonttheme{serif}

\setbeamertemplate{footline}[page number]{} 

\setbeamercolor{headline}{fg=blue!90!black,bg=cyan!90!black}
\setbeamercolor{palette primary}{fg=white,bg=cyan!90!black}
\setbeamercolor{palette secondary}{fg=white,bg=cyan!90!black}
\setbeamercolor{palette tertiary}{fg=white,bg=black}
\setbeamercolor{frametitle}{fg=white,bg=cyan!50!black}
\setbeamercolor{alerted text}{fg=brightpink}


\colorlet{titleleft}{cyan}
\colorlet{titleright}{black}

\setbeamercolor*{frametitle}{fg=white}

\makeatletter
\pgfdeclarehorizontalshading[titleleft,titleright]{beamer@frametitleshade}{\paperheight}{%
    color(0pt)=(titleleft);
    color(\paperwidth)=(titleright)}
\makeatother

%============================ continue enumerate ===================================
% \setbeamercovered{highly dynamic}

\newcounter{saveenumi}
\newcommand{\seti}{\setcounter{saveenumi}{\value{enumi}}}
\newcommand{\conti}{\setcounter{enumi}{\value{saveenumi}}}

%====================================================================================

% ====================================================================================================

% =================== Different colors for different blocks =======================================
% Thanks to https://tex.stackexchange.com/questions/347269/different-colors-for-different-types-of-blocks-in-beamer

\setbeamercolor{block title}{use=structure,fg=structure.fg,bg=structure.fg!20!bg}
% \setbeamercolor{block body}{parent=normal text,use=block title,bg=block title.bg!50!bg}

\setbeamercolor{block title example}{use=example text,fg=example text.fg,bg=example text.fg!20!bg}
% \setbeamercolor{block body example}{parent=normal text,use=block title example,bg=block title example.bg!50!bg}

\addtobeamertemplate{proof begin}{%
    \setbeamercolor{block title}{fg=black,bg=yellow!50!white}
    % \setbeamercolor{block body}{fg=red, bg=red!30!white}
}{}

\BeforeBeginEnvironment{theorem}{
    \setbeamercolor{block title}{fg=black,bg=orange!50!white}
    % \setbeamercolor{block body}{fg=orange, bg=orange!30!white}
}
\AfterEndEnvironment{theorem}{
 \setbeamercolor{block title}{use=structure,fg=structure.fg,bg=structure.fg!20!bg}
 % \setbeamercolor{block body}{parent=normal text,use=block title,bg=block title.bg!50!bg, fg=black}
}

\BeforeBeginEnvironment{definition}{%
    \setbeamercolor{block title}{fg=black,bg=cyan!50!white}
    % \setbeamercolor{block body}{fg=pink, bg=pink!30!white}
}
\AfterEndEnvironment{definition}{
 \setbeamercolor{block title}{use=structure,fg=structure.fg,bg=structure.fg!20!bg}
 % \setbeamercolor{block body}{parent=normal text,use=block title,bg=block title.bg!50!bg, fg=black}
}

\BeforeBeginEnvironment{remark}{%
    \setbeamercolor{block title}{fg=black,bg=teal!50!white}
    % \setbeamercolor{block body}{fg=pink, bg=pink!30!white}
}
\AfterEndEnvironment{remark}{
 \setbeamercolor{block title}{use=structure,fg=structure.fg,bg=structure.fg!20!bg}
 % \setbeamercolor{block body}{parent=normal text,use=block title,bg=block title.bg!50!bg, fg=black}
}
% ====================================================================================================


\newcommand{\rbb}{\ensuremath{\mathbb{R}}}
\newcommand{\qbb}{\ensuremath{\mathbb{Q}}}
\newcommand{\zbb}{\ensuremath{\mathbb{Z}}}
\newcommand{\nbb}{\ensuremath{\mathbb{N}}}
\newcommand{\vectorsall}{\ensuremath{\mathbf{v}_1,\mathbf{v}_2,\ldots,\mathbf{v}_n}} 
\newcommand{\vectorsallw}{\ensuremath{\mathbf{w}_1,\mathbf{w}_2,\ldots,\mathbf{w}_n}} 
\newcommand{\xbf}{\ensuremath{\mathbf{x}}} 
\newcommand{\ybf}{\ensuremath{\mathbf{y}}} 
\newcommand{\zbf}{\ensuremath{\mathbf{z}}} 
\newcommand{\vbf}{\ensuremath{\mathbf{v}}}
\renewcommand{\bf}[1]{\ensuremath{\mathbf{#1}}}
\newcommand{\sbb}{\ensuremath{\mathbb{S}}}
\newcommand{\cali}[1]{\ensuremath{\mathcal{#1}}}
\newcommand{\scr}[1]{\ensuremath{\mathscr{#1}}}
\newcommand{\mscr}{\ensuremath{\mathscr{M}}}
\newcommand{\nscr}{\ensuremath{\mathscr{N}}}
\renewcommand{\sf}[1]{\ensuremath{\mathsf{#1}}}
\newcommand{\red}[1]{\textcolor{red}{#1}} 
\newcommand{\blue}[1]{\textcolor{blue}{#1}} 
\newcommand{\brown}[1]{\textcolor{brown}{#1}} 
\newcommand{\magenta}[1]{\textcolor{magenta}{#1}}
\definecolor{defn}{rgb}{0.9, 0.17, 0.31}
\newcommand{\defn}[1]{\textcolor{defn}{#1}}
\newcommand{\cyan}[1]{\textcolor{cyan}{#1}}
\newcommand{\bb}[1]{\ensuremath{\mathbb{#1}}} 
\newcommand{\delbydel}[2]{\ensuremath{\dfrac{\partial#1}{\partial#2}}} 
\newcommand{\crf}{\ensuremath{\mathrm{Cr}(f)}} 
\newcommand{\grad}{\ensuremath{\nabla}} 
\newcommand{\dbb}{\ensuremath{\mathbb{D}}} 
\newcommand{\dist}{\ensuremath{\operatorname{dist}}} 
\newcommand{\dbyd}[2]{\ensuremath{\dfrac{d#1}{d#2}}} 
\newcommand{\isom}{\ensuremath{\cong}} 
\renewcommand{\cal}[1]{\ensuremath{\mathcal{#1}}} 
\newcommand{\acal}{\ensuremath{\cal{A}}}
\newcommand{\im}{\ensuremath{\operatorname{im}}}
\newcommand{\defeq}{\vcentcolon=}
\newcommand{\eqdef}{=\vcentcolon}
\newcommand{\define}{\overset{\mathrm{def}}{=\joinrel=}}
\newcommand{\tensor}{\ensuremath{\otimes}}
\newcommand{\directsum}{\ensuremath{\oplus}}
\newcommand{\homeo}{\ensuremath{\cong}}
\newcommand{\cutn}{\ensuremath{\mathrm{Cu}(N)}}
\newcommand{\sen}{\ensuremath{\mathrm{Se}(N)}}
\newcommand{\innerprod}[2]{\ensuremath{\left\langle #1,#2\right\rangle}}
\newcommand{\norm}[1]{\ensuremath{\left\|#1\right\|}}
\newcommand{\range}{\ensuremath{\mathrm{range}}}
\newcommand{\paran}[1]{\ensuremath{\left( #1 \right)}}
\newcommand{\curlybracket}[1]{\ensuremath{\left\{ #1 \right\}}}
\newcommand{\squarebracket}[1]{\ensuremath{\left[ #1 \right]}}
\newcommand{\abs}[1]{\ensuremath{\left|#1\right|}}
\newcommand{\aTransInverse}{\ensuremath{\paran{A^T}^{-1}}}
\newcommand{\sqrtATransAInverse}{\ensuremath{\paran{\sqrt{A^TA}}^{-1}}}
\newcommand{\trace}[1]{\ensuremath{\mathrm{tr}\left( #1 \right)}}
\newcommand{\cu}{\ensuremath{\mathrm{Cu}(p)~}}
\newcommand{\se}{\ensuremath{\mathrm{Se}(p)~}} 
\newcommand{\co}{\ensuremath{\mathrm{Co}(x_0,\delta)~}} 
\newcommand{\costar}{\ensuremath{\mathrm{Co}^\star(x_0,\delta)~}}
\newcommand{\Ball}{\ensuremath{\overline{B(x_0,\delta)}~}}
\newcommand{\scal}{\mathcal{S}}
\newcommand{\bcal}{\mathcal{B}}
\newcommand{\hess}{\mathrm{Hess}}

% New commands for todo notes

\newcommand{\lub}{\ensuremath{\mathrm{lub}}}
\theoremstyle{definition} 
\newtheorem*{remark}{Remark}
\newtheorem*{coro}{Corollary}
\newtheorem*{prop}{Proposition}
\newcommand{\p}{\pause}

% Newcommands fo r font size
\newcommand{\fontsmall}{\fontsize{7pt}{8.2}\selectfont}


\usepackage{amsmath, amssymb, amscd, amsthm, amsfonts, amsbsy, appendix}
\usepackage{lmodern}
\usepackage{bbm, mathrsfs, bm, linearb, upgreek, textcomp, mathdots}
\usepackage{mathtools, url}
\usepackage[all, cmtip]{xy} 
\usepackage[intoc]{nomencl}
\usepackage{pgf}
\usepackage{calligra}
\usepackage{ebgaramond}
\usepackage{bookmark}% faster updated bookmarks
\usepackage[most]{tcolorbox}
\usepackage{varwidth}   %% provides varwidth environment

\usepackage{tikz,tcolorbox}
\usetikzlibrary{cd}
\usetikzlibrary{shapes.callouts}

\usepackage{etoolbox}

\makeatletter
\newcommand{\DrawLine}{%
  \begin{tikzpicture}
  \path[use as bounding box] (0,0) -- (\linewidth,0);
  \draw[color=red!50!yellow,dashed,dash phase=2pt]
        (0-\kvtcb@leftlower-\kvtcb@boxsep,0)--
        (\linewidth+\kvtcb@rightlower+\kvtcb@boxsep,0);
  \end{tikzpicture}%
  }
\makeatother

% For pause in align
%https://tex.stackexchange.com/questions/16186/equation-numbering-problems-in-amsmath-environments-with-pause/75550#75550
\makeatletter
\let\save@measuring@true\measuring@true
\def\measuring@true{%
  \save@measuring@true
  \def\beamer@sortzero##1{\beamer@ifnextcharospec{\beamer@sortzeroread{##1}}{}}%
  \def\beamer@sortzeroread##1<##2>{}%
  \def\beamer@finalnospec{}%
}

% For continued proof 

\newenvironment<>{proofs}[1][\proofname]{%
    \par
    \def\insertproofname{#1\@addpunct{.}}%
    \usebeamertemplate{proof begin}#2}
  {\usebeamertemplate{proof end}}
\makeatother

% \hypersetup{pdfpagemode=FullScreen}




\AtBeginSection[]{
  \begin{frame}
  \vfill
  \centering
  \begin{beamercolorbox}[sep=8pt,center,shadow=true,rounded=true]{title}
    \usebeamerfont{title}\secname\par%
  \end{beamercolorbox}
  \vfill
  \end{frame}
}

%\hypersetup{pdfpagemode=FullScreen}
\title[Cut Locus]{Morse-Bott functions, Cut locus and their relations \\[1ex] \small DMV-{\"O}MV Jahrestangung 2021}
\author{Sachchidanand Prasad}
\institute[IISERK]{\small Indian Insitute of Science Education and Research Kolkata}
\date{1st October, 2021}
% logo 
\titlegraphic{%
   \includegraphics[width=1.5cm,keepaspectratio]{figures/iiserk-logo.png}%
}

% \logo{%
%   \makebox[0.95\paperwidth]{%
%     % \includegraphics[width=4cm,keepaspectratio]{img/csm_dmv-logo.png}%
%     \hfill%
%     \includegraphics[width=1.5cm,keepaspectratio]{img/iiserk-logo.png}%
%   }%
% }

% \definecolor{titlepage}{rgb}{0.98, 0.94, 0.75}

\setbeamertemplate{navigation symbols}{} % Navigation symbols will be removed
% \setbeamertemplate{footline}{}
\setbeamertemplate{itemize items}[ball]  % Itemize into balls

\begin{document}

\begin{frame}
 \titlepage
  \vspace{0.1cm}
  \begin{center}
    \Large \textcolor[rgb]{0.4,0,0.5}{Young Topologists and Geometers} \\
  \end{center}
\end{frame}

	\begin{frame}
		\frametitle<presentation>{Outline of the talk}
		\tableofcontents
	\end{frame}	

\section{Geometric aspects of the cut locus}

	\begin{frame}
		\frametitle<presentation>{Morse-Bott Function}
		
		\p\begin{definition}[Morse-Bott functions]
			\p Let $M$ be a Riemannian manifold. A smooth submanifold $ N\subset M $ is said to be {\emph{\defn{non-degenerate critical submanifold}}} of $f:M\to \mathbb{R}$ if $N\subseteq \crf$ \p and for any $p\in N$, $\hess_{p}(f)$ is \alert<5>{non-degenerate in the direction normal to $N$ at $p$}. \p[6] The function $f$ is said to be {\emph{\defn{Morse-Bott}}} if the connected components of $ \crf $ are non-degenerate critical submanifolds.
		\end{definition}
		\p[5] The $\hess_p(f)$ is \alert<5>{non-degenerate in the direction normal to $N$ at $p$} means for any $V\in (T_pN)^\perp$ there exists $W\in  (T_pN)^\perp$ such that $\hess_p(f)(V,W)\neq 0$.
	\end{frame}	

\begin{frame}
		\frametitle<presentation>{Cut locus of a submanifold}

		\p
		\begin{definition}[Distance minimal geodesic] \label{defn:distance_minimal_geodesic}
			\p A geodesic $\gamma $ is called a \emph{\defn{distance minimal geodesic}} joining $N$ to $p$ \p if there exists $q\in N$ \p such that $\gamma$ is a minimal geodesic joining $q$ to $p$ \p and $l(\gamma)= d(p,N) $. \p We will call such geodesics as \textit{$N$-geodesics}.
		\end{definition}

		\p 
		\begin{definition}[Cut locus] \label{defn:cut_locus}
			\p Let $M$ be a Riemannian manifold \p and $N$ be any non-empty subset of $M$. \p If $\cutn$ denotes the \emph{\defn{cut locus of $N$}}, \p then we say that $q\in \cutn $ if there exists an $N$-geodesic joining $N$ to $q$ \p such that any extension of it beyond $q$ is not a distance minimal geodesic.
		\end{definition}
	\end{frame}

	\begin{frame}
		\frametitle<presentation>{An Example}
		\begin{figure}[htbp]
			\begin{overlayarea}{6cm}{6cm}
				\includegraphics<2>[page = 1, width = \textwidth]{figures/cut-locus-of-great-circle.pdf}
				\includegraphics<3>[page = 2, width = \textwidth]{figures/cut-locus-of-great-circle.pdf}
				\includegraphics<4>[page = 3, width = \textwidth]{figures/cut-locus-of-great-circle.pdf}
				\includegraphics<5>[page = 4, width = \textwidth]{figures/cut-locus-of-great-circle.pdf}
				\includegraphics<6>[page = 5, width = \textwidth]{figures/cut-locus-of-great-circle.pdf}
			\end{overlayarea}
		\end{figure}
	\end{frame}	
	

	\begin{frame}
		\frametitle<presentation>{Separating set of \texorpdfstring{$N$}{N}}

		\p 
		\begin{definition}[Separating set] \label{defn:separating_set}
			\p Let $N$ be a subset of a Riemannian manifold $M$. \p The \emph{\defn{separating set}}, denoted by $\sen$, \p  consists of all points $q\in M$ \p such that at least two distance minimal geodesics from $N$ to $q$ exist.
		\end{definition}

		\p 
		\begin{theorem}[Basu S., Prasad S., 2021]\label{thm:se_closure_is_cu}
			For a complete Riemannian manifold $M$ and a compact submanifold $N$ of $M$, 
			\begin{displaymath}
				\overline{\sen} = \cutn.
			\end{displaymath}
		\end{theorem}
	\end{frame}

% \section{An illuminating example}
	\begin{frame}
		\frametitle<presentation>{An illuminating example}
		\p Let $ M = M(n,\rbb) $, the set of $n\times n$ matrices, and $ N=O(n,\rbb) $, set of all orthogonal $n\times n$ matrices. \p We fix the standard Euclidean metric on $M(n,\rbb)$ by identifying it with $\mathbb{R}^{n^2}$. \p This induces a distance function given by
		\begin{equation*}
			d(A,B) \defeq \sqrt{\trace{(A-B)^T(A-B)}}, ~~A,B\in M(n,\rbb)
		\end{equation*}
		\p
		Consider the distance squared function
		\begin{displaymath}
		f: M(n,\rbb)\to \rbb,~~ A\mapsto d^2(A, O(n,\rbb)).
		\end{displaymath}
	\end{frame}

	\begin{frame}
	\setbeamercovered{transparent}
		\begin{itemize}[<+-| alert@+>]
			   \item The function is $f(A) = n + \trace{A^TA} - 2\trace{\sqrt{A^TA}}$.
			   \item It is differentiable at $A$ if and only if $A$ is invertible.
			   \item It is a Morse-Bott function with critical submanifold as $O(n,\rbb)$.
			   \item If $\gamma(t)$ is an integral curve of $-\grad f$ initialized at $A$, then 
	   			\begin{equation}
	   				\dfrac{d\gamma}{dt} = -2\gamma(t)+2\paran{\gamma(t)^T}^{-1}\sqrt{\gamma(t)^T\gamma(t)} \label{eq: 4.5}.
	   			\end{equation}
			    \item The solution of \eqref{eq: 4.5} given by
			
			\begin{equation}\label{eq: 4.6}
			\gamma(t) = Ae^{-2t} + (1-e^{-2t})A\paran{\sqrt{A^TA}}^{-1},~~\gamma(0)=A.
			\end{equation} 
			   \item Note that $\gamma(t)$ is a flow line which deforms $GL(n,\rbb)$ to $O(n,\rbb)$.
			   \item The separating set of $O(n,\rbb)$ in $M(n,\rbb)$ is set of singular matrices and as it is closed, the cut locus is the same.
		\end{itemize}
		% \setbeamercovered{transparent}
	\end{frame}	

	\begin{frame}
		\frametitle<presentation>{Results generalized from the example}
		\begin{theorem}
			\p Let $M$ be a connected, \p complete Riemannian manifold \p and $N$ be an embedded submanifold of $M$. \p Suppose two  $N$-geodesics exists joining $N$ to $q\in M$ \p Then $d^2(N,\cdot):M\to \mathbb{R}$ has no directional derivative at $q~$ \p for vectors in direction of those two $N$-geodesic. 
		\end{theorem}
		
		\p 
		\begin{theorem}
			Let $M$ be a complete Riemannian manifold \p and $N$ be compact submanifold of $M$. \p Then $N$ is a deformation retract of $M-\cutn$.
		\end{theorem}

		\p 
		\begin{theorem}
			The cut locus $\cutn$ is a strong deformation retract of $M-N$. \p In particular, $(M,\cutn)$ is a good pair \p and the number of path components of $\cutn$ equals that of $M-N$.
		\end{theorem}
	\end{frame}	

	\begin{frame}
		\frametitle<presentation>{Outline of the proof of the deformation}
		\p Define
			\begin{displaymath}
				\mathbf{s}:S(\nu)\to [0,\infty],\,\,\mathbf{s}(v):=\sup\{t\in [0,\infty)\,|\,\gamma_v |_{[0,t]}\,\,\textup{is an $N$-geodesic}\},
			\end{displaymath}
			where $S(\nu)$ is the unit normal bundle of $N$ and $[0,\infty]$ is the one-point compactification of $[0,\infty)$. \p The map $\mathbf{s}$ is continuous \p and is finite if $M$ is compact. \p Note that the cut locus is
			\begin{displaymath}
				\cutn = \exp_\nu\curlybracket{\mathbf{s}(v)v: v\in S(\nu)},
			\end{displaymath}
			\p where $\exp_\nu:\nu\to M,~\exp_\nu(p,v)\defeq \exp_p(v)$.
			\p Define an open neighborhood $U_0(N)$ of the zero section in the normal bundle as
			\begin{displaymath}
				U_0(N) \defeq \curlybracket{av: 0\le a< {\mathbf{s}}(v),~v\in S(\nu)}.
			\end{displaymath}
			\p Note that $\exp_\nu$ is a diffeomorphism on $U_0(N)$ and set $U(N) = \exp_\nu(U_0(N)) = M - \cutn$. 
	\end{frame}	
	\begin{frame}
		\frametitle<presentation>{}
		\begin{figure}[htbp]
			\centering
			\includegraphics<1>[width=0.95\textwidth]{figures/normal_bundle-top_seminar-2.pdf}
			\includegraphics<2>[width=0.95\textwidth]{figures/normal_bundle-top_seminar-3.pdf}
			\includegraphics<3>[width=0.95\textwidth]{figures/normal_bundle-top_seminar-4.pdf}
			\includegraphics<4->[width=0.95\textwidth]{figures/normal_bundle-top_seminar-5.pdf}
		\end{figure}
		\p[5] The space $U_0(N)$ deforms to the zero section on the normal bundle. 
		\p[6]  
		\begin{align*}
			H : U_0(N) \times [0,1] \to U_0(N), ((p,av),t)\mapsto (p,tav).
		\end{align*}
	\end{frame}	

	\begin{frame}
		Now consider the following diagram:
		\begin{align*}
			\xymatrix{
				U_0(N) \times [0,1] \ar[r]^H  				   & U_0(N) \ar[d]^{exp_\nu} \\  
				U \times [0,1] \ar[u]^{\exp_\nu^{-1}} \ar[r]^F & U \only<1->{\cong M-\cutn }
			}
		\end{align*}
		\p[2] 
		The map $F$ can be defined by taking the compositions
		\begin{displaymath}
			F = \exp_\nu \circ H \circ \exp_\nu^{-1}.
		\end{displaymath}
	\end{frame}	


	\section{Topological aspects of the cut locus}
	\begin{frame}
		\frametitle<presentation>{Thom space}
		 
		 \p
		 \begin{definition}[Thom space]
		 	Let  $\pi: E\to B$ be a real vector bundle over a paracompact space $B$ with a metric. Let $D(E)$ be the unit disk bundle and $S(E)$ be the unit sphere bundle. \p Then the \emph{\defn{Thom space of $E$}}, denote by $\mathrm{Th}(E)$ is the quotient $\mathrm{Th}(E) \defeq D(E)/S(E)$.
		 \end{definition}

		 \p
		 \begin{remark}
		 	If $B$ is compact, then $\mathrm{Th}(E)$ is the one point compactification of $E$. 
		 \end{remark}
	\end{frame}	


	\begin{frame}
		\begin{definition}[Rescaled exponential]
			The \emph{\defn{rescaled exponential}} map is defined to be 
			\begin{displaymath}
				\widetilde{\exp}:D(\nu)\to M,~ (p,v)\mapsto \begin{cases}
					\exp_p(\mathbf{s}(\hat{v})v), &\text{ if } v = \|v\|\hat{v}\\
					p, & \text{ if } v=0.
				\end{cases}
			\end{displaymath}
		\end{definition}

		\vspace{1.5cm}
		\p
		\begin{remark}
			Since $\mathbf{s}$ is continuous, the rescaled exponential is also continuous and is surjective. \p Also note that $\widetilde{\exp}(S(\nu)) = \mathrm{Cu}(N)$.
		\end{remark}
	\end{frame}	

	\begin{frame}
		\frametitle<presentation>{The Main Theorem}
		\begin{theorem}[Basu S., Prasad S., 2021]
			Let $N$ be an embedded submanifold inside a closed, connected Riemannian manifold $M$. If $\nu$ denotes the normal bundle of $N$ in $M$, then there is a homeomorphism 
			\begin{displaymath}
				\widetilde{\exp} : D(\nu)/S(\nu) \xrightarrow{\cong} M/\mathrm{Cu}(N).
			\end{displaymath}
		\end{theorem}
	\end{frame}	

	\begin{frame}
		\frametitle<presentation>{Applications}
		\begin{enumerate}
			\p \item The inclusion map $i:\mathrm{Cu}(N)\hookrightarrow M$  induces a long exact sequence in homology \p
			\begin{displaymath}
				\cdots \to H_j(\mathrm{Cu}(N)) \xrightarrow{i_*}H_j(M) \to H_j(M,\mathrm{Cu}(N)) \xrightarrow{\partial} H_{j-1}(\cutn) \to \cdots,
			\end{displaymath}
			\p so 
			\begin{displaymath}
				\cdots \to H_j(\mathrm{Cu}(N)) \xrightarrow{i_*}H_j(M) \xrightarrow{q} \tilde{H}_j(\mathrm{Th}(\nu)) \xrightarrow{\partial} H_{j-1}(\cutn) \to \cdots 
			\end{displaymath}
			\p \item If $N$ is a closed submanifold of $M$ with $l$ components, and $\dim M =d$, \p then $H_{d-1}(\cutn)$ is free abelian of rank $l-1$ and $H_{d-j}(\cutn)\equiv H^j(M)$ if $j-2\ge k$, where $k$ is the maximum of the dimension of the components of $N$.

			\p \item Let $N$ be a smooth homology $k$-sphere, $k>0$, embedded in a smooth Riemannian manifold $M$ homeomorphic to $S^d$. \p  If $d\ge k+3$, then the cut locus $\cutn$ is homotopy equivalent to $S^{d-k-1}$.
			\seti
		\end{enumerate}
	\end{frame}	

	\begin{frame}
		\begin{enumerate}
			\conti
			\item Let $N$ be a real analytic homology $k$-sphere embedded in a real analytic homology $d$-sphere $M$. If $d\ge k+3$, then the cut locus $\cutn$ is a simplicial complex of dimension at most $(d-1)$, having the homology of $(d-k-1)$-sphere with fundamental group isomorphic to that of $M$.
		\end{enumerate}
	\end{frame}	

	\begin{frame}
		\huge{
		\begin{center}
			Thank You for your attention!
		\end{center}
		}
	\end{frame}	

\end{document}